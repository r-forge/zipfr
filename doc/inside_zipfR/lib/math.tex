%%
%%  INCLUDE: math.tex
%%  
%%  basic mathematical symbols and constructs (not specific to cooccurrences)
%%


%% \setN, \setN[0], \setZ, \setQ, \setR, \setC
%% abbreviations for common number spaces
\newcommand{\setN}[1][]{\mathbb{N}_{#1}} % allows \setN and \setN[0]
\newcommand{\setZ}{\mathbb{Z}}
\newcommand{\setQ}{\mathbb{Q}}
\newcommand{\setR}{\mathbb{R}}
\newcommand{\setC}{\mathbb{C}}

%% \set{el_1, el_2, ...};  \setdef{el}{condition};  \bigset{..}, \bigsetdef{..}{..}
%% extensional and intensional definition of sets, with "big" versions (like \bigl etc.)
\newcommand{\set}[1]{\left\{#1\right\}}
\newcommand{\setdef}[2]{\set{#1\,\left|\,#2\right.}}
\newcommand{\bigset}[1]{\bigl\{#1\bigr\}}
\newcommand{\bigsetdef}[2]{\bigset{#1\bigm|#2}}
\newcommand{\Bigset}[1]{\Bigl\{#1\Bigr\}}
\newcommand{\Bigsetdef}[2]{\Bigset{#1\Bigm|#2}}
\newcommand{\biggset}[1]{\biggl\{#1\biggr\}}
\newcommand{\biggsetdef}[2]{\biggset{#1\biggm|#2}}

%% \compl{X} = complement of set X
\newcommand{\compl}[1]{\mathcal{C} #1}

%% \eps == \epsilon, \si == \sigma, \sisi == \sigma^2, \ka == \kappa
\newcommand{\eps}{\epsilon}
\newcommand{\si}{\sigma}
\newcommand{\sisi}{\sigma^2}
\newcommand{\ka}{\kappa}

%% \abs{expr}, \bigabs{expr}, \norm{expr}, \bignorm{expr}
%% absolute value and norm of expression, with "big" versions
\newcommand{\abs}[1]{\left\lvert#1\right\rvert}
\newcommand{\bigabs}[1]{\bigl\lvert#1\bigr\rvert}
\newcommand{\norm}[1]{\left\lVert#1\right\rVert}
\newcommand{\bignorm}[1]{\bigl\lVert#1\bigr\rVert}

%% \constpi == constant PI (in bold font)
\newcommand{\constpi}{\boldsymbol{\pi}}

%% \dx == "dx";  \dx[z] == "dz";  \dpi == \dx[\pi];  
%% \dG == \dx[G], \dt == \dx[t]
\newcommand{\dx}[1][x]{\,d#1}
\newcommand{\dpi}{\dx[\pi]}
\newcommand{\dG}{\dx[G]}
\newcommand{\dt}{\dx[t]}

%% \Int{\frac{1}{2} x^2}_a^b
%% anti-derivative evaluated to compute definite integral
\newcommand{\Int}[1]{\left[#1\right]}

%% \limdownto{x}{0}
%% limit from above for x -> 0
\newcommand{\limdownto}[2]{\lim_{#1\,\downarrow\,#2}}

%% \iffdef == ":<=>";  \iffdefR == "<=>:"
\newcommand{\iffdef}{\;:\!\iff}
\newcommand{\iffdefR}{\iff\!:\;}

%% \logten(x) 
%% base 10 logarithm, which is always used in the UCS system
\newcommand{\logten}{\log_{10}}

%% \e+3, \e-6, \e-{12}, 5.5\x\e-3
%% engineering-style notation (orders of magnitude) for floating-point numbers
\newcommand{\e}[2]{10^{\ifthenelse{\equal{#1}{+}}{}{#1}#2}}
\newcommand{\x}{\cdot}

%% \Landau{ n^2 }, \bigLandau{ N^2 }
%% Landau symbol ("big oh notation")
\newcommand{\Landau}[1]{\mathcal{O}\left({#1}\right)}
\newcommand{\bigLandau}[1]{\mathcal{O}\bigl({#1}\bigr)}


%%% Local Variables: 
%%% mode: latex
%%% TeX-master: t
%%% End: 
